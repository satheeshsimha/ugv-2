\documentclass[12pt]{article}
\usepackage{graphicx}
%\documentclass[journal,12pt,twocolumn]{IEEEtran}
\usepackage[none]{hyphenat}
\usepackage{graphicx}
\usepackage{listings}
\usepackage[english]{babel}
\usepackage{graphicx}
\usepackage{caption} 
\usepackage{hyperref}
\usepackage{booktabs}
\usepackage{commath}
\usepackage{gensymb}
\usepackage{array}
\usepackage{amsmath}   % for having text in math mode
\usepackage{listings}
\lstset{
  frame=single,
  breaklines=true
}
  
%Following 2 lines were added to remove the blank page at the beginning
\usepackage{atbegshi}% http://ctan.org/pkg/atbegshi
\AtBeginDocument{\AtBeginShipoutNext{\AtBeginShipoutDiscard}}
%


%New macro definitions
\newcommand{\mydet}[1]{\ensuremath{\begin{vmatrix}#1\end{vmatrix}}}
\providecommand{\brak}[1]{\ensuremath{\left(#1\right)}}
\providecommand{\norm}[1]{\left\lVert#1\right\rVert}
\newcommand{\solution}{\noindent \textbf{Solution: }}
\newcommand{\myvec}[1]{\ensuremath{\begin{pmatrix}#1\end{pmatrix}}}
\let\vec\mathbf


\begin{document}
\begin{center}
\title{\textbf{Properties of vectors}}
\date{\vspace{-5ex}} %Not to print date automatically
\maketitle
\end{center}
\setcounter{page}{1}
\section{12$^{th}$ Maths - Exercise 10.4.1}

\begin{enumerate}
\item Find $\abs{\overrightarrow{a}\times\overrightarrow{b}}\text{ if }\overrightarrow{a}=\hat{i}-7\hat{j}+7\hat{k}\text{ and }\overrightarrow{b}=3\hat{i}-2\hat{j}+2\hat{k}$
\section{Solution}
Now,
\begin{align}
\text{Let } \vec{A} = \myvec{1\\-7\\7} \text{ and } \vec{B} = \myvec{3\\ -2 \\ 2}\\
\end{align}
The cross product or vector product of $\vec{A},\vec{B}$ is defined as
\begin{align}
	\vec{A} \times \vec{B} = \myvec{\mydet{\vec{A}_{23}&\vec{B}_{23}\\\vec{A}_{31}&\vec{B}_{31}\\\vec{A}_{12}&\vec{B}_{12}}}
\end{align}
Hence
\begin{align}
	\mydet{\vec{A}_{23}&\vec{B}_{23}}&=\mydet{-7&-2\\7&2}=\myvec{-14+14}=0\\
	\mydet{\vec{A}_{31}&\vec{B}_{31}}&=\mydet{1&3\\7&2}=\myvec{2-21}=-19\\
	\mydet{\vec{A}_{12}&\vec{B}_{12}}&=\mydet{1&3\\-7&-2}=\myvec{-2+21}=19\\
\end{align}

which can be represented in matrix form as
\begin{align}
	\vec{A} \times \vec{B}&=\myvec{0\\-19\\19}.
\end{align}
Hence
\begin{align}
\vec{\overrightarrow{A} \times \overrightarrow{B}}&=\sqrt{0^2+(-19^2)+19^2}\\
&=\sqrt{0+361+361}\\
 &= \sqrt{722}\\
 &=26.87
\end{align}

\end{enumerate}
\end{document}